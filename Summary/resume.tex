% !TeX program = lualatex
% Using VimTeX, you need to reload the plugin (\lx) after having saved the document in order to use LuaLaTeX (thanks to the line above)

\documentclass[a4paper]{article}

% Expanded on 2025-06-02 at 17:38:31.


\usepackage{../../style}

\title{Résumé AICC II}
\author{Arthur Herbette}
\date{Lundi 02 juin 2025}

\begin{document}
\maketitle
\section{Théorème est définition}

\begin{definition}
Un \important{code de source}, ou \important{encodage}, $\Gamma$ est une application bijective $\Gamma: \mathcal{A} \to \mathcal{C}$.
\end{definition}
\begin{definition}
Soit $\Gamma$ un code de source. $\Gamma$ est \important{à décodage unique} si pur toute suite de symbols de code, qui résulte de l'encodage d'une suite de symbole de la source, il existe un décodage unique.
\end{definition}
\begin{definition}
Nous disons qu'un code est \important{instantané}
\begin{itemize}
    \item s'il est à décodage unique
    \item et si à mesure que les séquences de symboles de l'alphabet du code sont reçus, les mots du code peuvent être déterminés sans s'inquiéter des symboles de code suivants.
\end{itemize}
\end{definition}
\begin{definition}
    On dit qu'un mot de code $c = x_1x_2\ldots x_k$ est \important{prefixe} d'un autre mot de code $c' $ si on peut écrire $c' = x_1x_2x_kx_{k+1}\ldots x_l$ pour un certain $ l \geq k + 1 $. Sur l'arbre du code, cela veut dire que $c'$ est un descendant de $c$. On dit que le code $\Gamma$ est \important{sans préfixe} si aucun mot de code n'est prefixe d'un autre mot de code. Sur l'arbre du code, cela veut dire que aucun mot de code n'est descendant d'un autre mot de code.
\end{definition}
\begin{theoreme}
Un code est sans préfixe si et seulement s'il est instantané
\end{theoreme}


\subsection{Kraft-MCMillan}
\begin{theoreme}
Soit $\Gamma$ un code D'aire dont les longueurs des $M$ mots de code sont $l_1, \ldots, l_M$. Si $\Gamma$ est à décodage unique alors il satisfait l'\important{inégalité de Kraft}:
\begin{align*}D^{-l_1} + \cdots + D^{-l_M} \leq 1 \end{align*}
Réciproquement, si des nombres $l_1, \ldots, l_M$ satisfont l'inégalité de Kraft, il existe un code D'aire instantané (donc à décodage unique) dont le dictionnaire possède $M$ mots de code et dont les longueurs des mots de code sont $l_1, \ldots, l_M$.
\end{theoreme}

\begin{theoreme}
Pour tout code à décodage unique, il existe un code instantané sur les mêmes alphabets de source et de code qui a les mêmes longueurs de mot
\end{theoreme}
\subsection{Efficacité d'un code de source}
\begin{definition}
Soit une source $S$ d'alphabet $\mathcal{A}$ et de densité de probabilité $p$, et soit $\Gamma$ un code D'aire de la source $S$. La \important{longueur moyenne} du code $\Gamma$ est:
\begin{align*} 
    L\left(\Gamma\right) = \sum_{s \in \mathcal{A}}p\left(s\right)l\left(\Gamma \left(s\right)\right)  
\end{align*}
L'unité est le symbole de code par symbole de source (si $D = 2$ on dit bits par symbole de source)
\end{definition} 
\begin{theoreme}
Soit une source $S$ d'entropie $H\left(S\right)$ et soit $\Gamma$ un code D'aire de la source $s$. Si $\Gamma$ est à décodage unique, sa longueur moyenne satisfait:
\begin{align*} L\left(\Gamma\right) \geq \frac{H\left(S\right)}{\log_2 \left(D\right)} \end{align*}
\end{theoreme}


\end{document}

