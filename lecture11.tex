\lecture{11}{2025-03-25}{Number Theory}{}
    \section{Number theory}
    \begin{parag}{Introduction}
        In the digital world, the information is represented by the elements of a finite set, and we should be able to do math with them. Which means that the finite set should be a finite field. Our bigger goal of the next few lectures is to develop the tools to understand when and how we can turn a finite set into a finite field.
    \end{parag}
    
    \begin{parag}{Operation with integers}
        Within $ \mathbb{Z}$ (the set of integers) we can
        \begin{itemize}
            \item add, subtract, multiply
            \item but not divide $ \frac{7}{2}$ is not an integer
            \item What comes closest to the (regular) division is the euclidean division
        \end{itemize}
    \end{parag}
    
    \begin{parag}{Euclidiean division}
        \begin{subparag}{Example}
            For example if we take $7$ divided by $2$ this is equal to $3.5$. (because $2 \cdot  3.5 = 7$)\\
            But if we take $25$ divided by $4 = 6.25$ but,
            \begin{align*} 25 \Mod 4 = 1 \end{align*}
            Which leads to $25 = 4 \cdot  6 + 1$.
            \begin{framedremark}
                the goal here is to find that future ``inverse'' in the integer world.
            \end{framedremark}

        \end{subparag}
        \begin{subparag}{The division algorithm}
            Given integers a (the dividend) and $m$ the divisor:
            \begin{align*}
                a = mq + r, \; \; 0 \leq r < \mid m \mid
            \end{align*}
            \begin{framedremark}
                The computation of $q$ and $r$ as above is called euclidean division
            \end{framedremark}
        \end{subparag}
     
        \begin{subparag}{euclidean division in mainstream programming languages}
            In  C/C++/Java/ Python we use the operator \% to compute $r$ as follows:\\
            if $a$ and $m$ are both positive, then $r = \% m$\\
            If one or the other or both are negative, different languages behave differently, but the general rule is:
            \begin{itemize}
                \item if $ a \% m$ is nonnegative, then $r  = a \% m$
                \item if $ a \% m$ is \important{negative}, then $r = a \% m + m$
            \end{itemize}

            
        \end{subparag}
    \end{parag}
    
    \begin{parag}{Congruence}
        Sometimes we are interested in knowing if two numbers have the same remainder when divided by $m$.
        \begin{definition}
            Two integers $a$ and $b$ are said to be \important{congruent modulo} $m$ , denoted:
            \begin{align*} a \equiv b \left(\Mod m\right) \end{align*}
            if $m \mid a - b$.
            (or $m$ divided $a - b$)
        \end{definition}
        \begin{subparag}{Note}
            Do not confuse the \important{relation} $a \equiv b \Mod m$ and the \important{function}  $a \to a \Mod m$.
        \end{subparag}
    \end{parag}
    \begin{parag}{Some laws that can be useful}
        \begin{itemize}
            \item $a \equiv b \Mod m$
            \item $\left(a-b\right) \Mod m = 0$
            \item $a \Mod m =  b \Mod m$
        \end{itemize}
    \end{parag}
    
    
    \begin{parag}{Congruence is an equivalence relation}
        A binary relation $ \sim$ on a set is an \important{equivalence relation} if and only if the following three axioms are satisfied:
        \begin{itemize}
            \item $a \sim a$ (reflexivity)
            \item if $ a \sim b$ then $ b \sim a$ (symmetry)
            \item if $ a \sim b$ and $b \sim c$ then $ a \sim c$ (transitivity)
        \end{itemize}
        Substriture $a \sim a$ with $ a \equiv a \mod m$ etc...,  to see that congruene is an equivalence relation\\
        One of the consequences is that we can form equivalence classes and we can work with one representative of each class (this will become useful later.)
    
    \end{parag}
    \begin{parag}{Equivalence classes}
        An equivalence relation $\sim$ breaks $A$ into disjoint sets, called \important{equivalence classes}. \\
        It is like every element that ``have the same relation'' as $a$:
        \begin{align*} \left[a\right] = \left\{x \in A \mid x \sim a\right\} \end{align*}
        This will be very useful to works with $ \mathbb{Z} / m \mathbb{Z}$.
    \end{parag}
    
    
    \begin{parag}{Modulo}
        \begin{theoreme}
            If:
            \begin{align*}
                a \equiv a' \mod m\\
                b \equiv b' \mod m 
            \end{align*}
            then:
            \begin{align*}
                a + b \equiv a' + b' \mod m\\
                ab \equiv a'b' \mod m\\
                a^n \equiv (a')^n \mod m
            \end{align*}
        \end{theoreme}
        In particular, if $a' = ( a \mod m)$ and $b' = b \mod m)$, then we obtain the following facts (useful is mon calculation):
        \begin{itemize}
            \item $(a + b) \equiv ((a \mod m) + (b \mod m)) \mod m$
            \item Hence:
                \begin{align*}
                    a + b \mod = (a \mod m) + (b \mod m)) \mod m
                \end{align*}
            \item $ab \equiv ((a \mod m) (b \mod m) \mod m$
            \item \begin{align*}
                ab \mod m = ((a \mod m)(b \mod m) \mod m
            \end{align*}
        \item $a^n \equiv (a \mod m)^n \mod m$
            \begin{align*}
                a^n \mod m = (a \mod m)^n \mod m
            \end{align*}
        \end{itemize}
        
        \begin{subparag}{Example}
            is $9^{1000} + 9^{10^6}$ divisible by $5$?\\
            We compute first:
            \begin{align*}
                9 \equiv -1 \mod 5
            \end{align*}
            Which gives us:
            \begin{align*}
                9^{1000} + 9^{10^6} \equiv (-1)^{1000} + (-1)^{10^6} \equiv 1 + 1 \equiv 2 \mod 5 
            \end{align*}
            Hence, $9^{1000} + 9^{10^6}$ is not divisible by $5$
        \end{subparag}
    \end{parag}
   





