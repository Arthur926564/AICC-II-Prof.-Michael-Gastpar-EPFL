
\lecture{13}{2025-04-01}{Multiplicative Inververse}{}

\begin{parag}{Why modular arithmetic}
    Modular arithmetic is the foundation of number theory, therefore we need number theory for cryptography and for channel coding.

\end{parag}
\begin{parag}{Introduction to $ \mathbb{Z}/ m \mathbb{Z}$}
    Instead of considering integers and congruences ( $\mod m$) and write equation like:
    \begin{align*}
        a + b \equiv c ( \mod m)
    \end{align*}
    We would like to make our life easier like this:
    \begin{align*}
        a + b = c
    \end{align*}
    This can be done, if we give a new meaning to $a$, $b$ and $c$. namely we make them the congruence classe  $[a]_m, [b]_m [c]_m$ and $[c]_m$.
    \begin{definition}
        Let $m > 1$ be an integer, called the modulus.\\
        The set of all integers congruent to $a (\mod m)$ is called the congruence class of $ a$ modulo $m$.\\
        it is denoted by $[a]_m$.
    \end{definition}
    
    \begin{framedremark}
        It is the same foundation as the one of finite field (corps fini) in linear algebra. (With prof. Sherer).
    \end{framedremark}
\end{parag}

\begin{parag}{Other definition}
    \begin{definition}
        The set of all congruence classes modulo $m$ is denoted by $ \mathbb{Z}/ m \mathbb{Z}$ (which is read $ \mathbb{Z} \mod m$.
    \end{definition}
   \begin{subparag}{Note}
       Some authors use the notation $ \mathbb{Z}_m$
   \end{subparag} 
\end{parag}


\begin{parag}{Example}
    Some more example to see how this works:\\
    if we have the class $[a]_m$ and:
    \begin{align*}
        a = mq + r, \text{ with } 0 \leq r \leq m - 1
    \end{align*}
    Then we have that:
    \begin{align*}
        [a]_m = [r]_m
    \end{align*}
    
    If we take for example $[-13]_9$ and $[5]_9$ we can see here that there are equal such that
    \begin{align*}
        [-13]_9 = [5]_9
    \end{align*}

\end{parag}

\begin{parag}{Sum}
    In $ \mathbb{Z}/ m \mathbb{Z}$ we define the sum and the product as follows:
    \begin{itemize}
        \item $[a]_m + [b]_m = [a + b]_m$
        \item $[a]_m[b]_m = [ab]_m$
    \end{itemize}
    The result is the same regardless the choice of representative. In fact:
    \begin{itemize}
        \item if we choose $[a + km]_m$ instead of $[a]_m$
        \item and $[b + lm]_m$ instead of $[b]_m$
        \item Then we obtain $[a + km]_m + [b + lm]_m = [a + km + b + lm]_m$ which is equal to $[a + b]_m$
    \end{itemize}

\end{parag}

\begin{parag}{Properties of $+$ in $ \mathbb{Z} / m \mathbb{Z}$}
    The sum has the following properties:
    \begin{itemize}
        \item $[a]_m + ( [b]_m + [c]_m) = ([a]_m + [b]_m) + [c]_m$
        \item There exists an additive identity, namely $[0]_m$:
            \begin{align*}
                [a]_m + [0]_m = [0]_m + [a]_m = [a]_m
            \end{align*}
        \item There exists an inverse with respect to addition: every $[a]_m$ has an inverse, denoted $[-a]_m$ such that:
            \begin{align*}
                [a]_m + [-a]_m = [-a]_m + [a]_m = [0]_m
            \end{align*}
        \item Commutativity
            \begin{align*}
                [a]_m + [b]_m = [b]_m + [a]_m
            \end{align*}
    \end{itemize}

\end{parag}
\begin{parag}{Properties of $ \times$ in $ \mathbb{Z} / m \mathbb{Z}$}
    The multiplication has the following properties:
    \begin{itemize}
        \item associativity
            \begin{align*}
                [a]_m([b]_m[c]_m) = ([a]_m[b]_m)[c]_m
            \end{align*}
        \item multiplicative identity, namely $[1]_m$:
            \begin{align*}
                [a]_m [1]_m = [1]_m[a]_m = [a]_m
            \end{align*}
        \item commutativity
            \begin{align*}
                [a]_m[b]_m = [b]_m[a]_m
            \end{align*}
            
    \end{itemize}

\end{parag}


\begin{parag}{Mixed properties}
    \begin{itemize}
        \item Distributivity:
            \begin{align*}
                [a]_m([b]_m + [c]_m) = [a]_m[b]_m + [a]_m[c]_m
            \end{align*}
    \end{itemize}

    \begin{subparag}{The notation $k[a]_m$ in $ \mathbb{Z}/ m \mathbb{Z}$}
        For an arbitrary positive integer $k$, $k[a]_m$ is a short hand for 
        \begin{align*}
            \underbrace{[a]_m + [a]_m + \cdots  + [a]_m}_{ \text{k times}}
        \end{align*}
  We can easily verify that:
  \begin{align*}
      k[a]_m = [ka]_m = [k]_m[a]_m
  \end{align*}
    \end{subparag}
\end{parag}
\begin{parag}{Multiplicative Inverse}
    Some elements of $ \mathbb{Z} / m \mathbb{Z} $ have the \important{multiplicative inverse}.\\
    The multiplicative inverse of $[a]_m$, if it exists is an element $[b]_m$ such that:
    \begin{align*}
        [a]_m[b]_m = [1]_m
    \end{align*}
    \begin{theoreme}
        The multiplicative inverse if it exits it is unique, and it is denoted by $([a]_m^{-1})$.
    \end{theoreme}
    Furthermore $(([a]_m)^{-1})^{-1} = [a]_m$
   \begin{subparag}{Proof}
       Suppose first that $ab = 1$ and $ac = 1$ (a has two inverse).\\
       The we now that $ab = ac$. If we multiply both side by $b$:
       \begin{align*}
           bab = bac
       \end{align*}
       However we know that $ab = 1 = ba$:
       \begin{align*}
           b \cdot 1 = c \cdot 1\\
           b = c
       \end{align*}
   \end{subparag} 
    

   
\end{parag}
