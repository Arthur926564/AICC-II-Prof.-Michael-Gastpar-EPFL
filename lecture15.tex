\lecture{15}{2025-04-08}{Commutative Groups}{}
\begin{parag}{What's next}
    After $\mathbb{Z} \ m \mathbb{Z}$ we could proceed in two directions:
    \begin{subparag}{Finite groups}
        Focus on finite groups, which are finite sets with one operation, like $\left( \mathbb{Z}/ m \mathbb{Z}, +\right)$ we do so now because we need them for cryptography.
    \end{subparag}
    
    \begin{subparag}{Finite field}
        Focus on finite field, which are finite sets with two operations, like $\left( \mathbb{Z}/ m \mathbb{Z}, +, \cdot\right)$ With the extra property that every non-zero element gas a multiplicative inverse. We do so later as we need finites fields for channel coding
    \end{subparag}
\end{parag}
\subsection{Commutative Group}
\begin{definition}
A \important{commutative group} (also called Abelian group) is a set $G$ endowed with a binary operation * that combines any two elements $a$ and $b$ to form another element denoted $a$ * $b$. The groupe operation * must satisfy the following five axioms:
\begin{itemize}
    \item (Closure) For all $a, b \in G$ , $a$ * (b * c) = (a * b ) * c
    \item (Associativity) For all $a, b \in G$ a *( b * c) = ( a * b ) * c
    \item (Identity element): There exists an element $e \in G$ such that for all $a \in G$, \important{a * e = e * a = e}
    \item (Inverse element) For all $a \in G$ there exists a $b \in G$ such that a * b = b * a = e
    \item (Commutativity) For all  $a, b \in G$ a * b = b * a
\end{itemize}

\end{definition}


\begin{parag}{ $\mathbb{Z}/ m \mathbb{Z}*$}
    To obtain a commutative group with modulo multiplication, we take only the elements of $ \mathbb{Z}/ m \mathbb{Z}$ that have multiplicative inverse. The resulting set is denoted $\mathbb{Z}/ m \mathbb{Z}*$
    \begin{theorem}
    For every integer $m > 1$ ( $ \mathbb{Z}/ m \mathbb{Z}^*m \cdot$) is a commutative group.
    \end{theorem}
    
\end{parag}
\subsubsection{Euler function}
\begin{definition}
    Euler's function $\phi\left(n\right)$ (also called Euler's totient function) is  the number of positive integers in $\{1, \dots, n\}$ that are relatively prime to $n$.
\end{definition}

\begin{parag}{Observation}
    Here we can see two main thing :
    \begin{itemize}
        \item $\phi\left(m\right)$ is the cardinality of $ \mathbb{Z}/ m \mathbb{Z}^*$
        \item if $p$ is prime, $\phi\left(p\right) = p-1$
        
    \end{itemize}
    
    
\end{parag}
\begin{parag}{The cartesian product of a commutative groupe is a commutative group}
    Recall of the axioms of a commutative group:
 \begin{itemize}
    \item (Closure) For all $a, b \in G$ , $a$ * (b * c) = (a * b ) * c
    \item (Associativity) For all $a, b \in G$ a *( b * c) = ( a * b ) * c
    \item (Identity element): There exists an element $e \in G$ such that for all $a \in G$, \important{a * e = e * a = e}
    \item (Inverse element) For all $a \in G$ there exists a $b \in G$ such that a * b = b * a = e
    \item (Commutativity) For all  $a, b \in G$ a * b = b * a
\end{itemize}

We can see the $\left(a_1, a_2\right) \in \left(G_1, op_1\right) \times \left(G_2, op_2\right)$ we see that this \important{is} a commutative group.
\end{parag}
\begin{parag}{Isomorphism}
    Some sets endowed with an operation might look different, buit they are actually the same once their elements are re-labeled.
    \begin{definition}
    Let (G, *) and (H, $\oplus$) be sets, each endowed with an arbitrary binary operation.\\
    an \important{isomorphism} from (G, *) to (H, $\oplus$) is a bijction $\psi: G \to H$ such that
    \begin{equation*} \psi\left(a  *\right) = \psi\left(a\right) \oplus \psi\left(b\right) \end{equation*}
    holds for all $a, b \in G$\\
    We say that (G, *) and (H $\oplus$) are \important{isomorphic} if there exists an isomorphism between them.
    \end{definition}
    Je me suis arreeter vers les slide 50
\end{parag}



