\section{Finite Fields and Vector spaces}
\lecture{22}{2025-05-07}{PPO midterm hurts}{}



\begin{parag}{Yesterday}
    
    \begin{subparag}{Example}
        Let $V =  \mathbb{F}_7^3$ and define $S =  \{\left(x_1, x_2, x_3\right): x_i \in \mathbb{F}_7 \text{ and } x_1 + 2x_2 + 3x_3 = 0\}$\\
        $S$ is a subspace of $V$. (Be sure that you see why)
    \end{subparag}
    Let us verify that $\vec{x}, \vec{y} \in S$:\\
    \begin{enumerate}
        \item is $a\vec{x} \in S$?
        \item Is $\vec{x} + \vec{y} \in S$?
    \end{enumerate}
    \begin{enumerate}
        \item $ax_1 + 2ax_2 + 3ax_3 = a\left(x_1 + 2x_2 + 2x_3\right) =  0$
        \item $\left(x_1 + y_1\right) + 2\left(x_2 + y_2\right) + 3\left(x_3 + y_3\right)$ We want to check if this is still in the subspace:
            \begin{align*} 
                x_1 + 2x_2 + 3x_3 + y_1 + 2y_2 + 3y_3 = 0
            \end{align*}
            Here the number $1, 2, 3$ are values in $\mathbb{F}_7$ (all the scalar are also in the field)
    \end{enumerate}
    Which leads that $S$ is a subspace.
\end{parag}
\begin{parag}{Some definitions }
    \begin{definition}
    A linéaire combinations of a \textbf{list} $\left(\vec{v}_1, \ldots, \vec{v}_n\right)$ of vectors in $V$ is a vector of the form $\sum_{i = 0}^{n} \lambda_i \vec{v}_i$ where $\lambda_1, \ldots, \lambda_n \in \mathbb{F}$.\\
    The set of all linear combinations of $\left(\vec{v}_1, \ldots, \vec{v}_n\right)$ is called the \textbf{span} of $\left(\vec{v}_1, \ldots, \vec{v}_n\right)$ denoted span$\left(\vec{v}_1, \ldots, \vec{v}_n\right)$.\\
if span $\left(\vec{v}_1, \ldots, \vec{v}_n\right) =  V$, we say that $\left(\vec{v}_1, \ldots, \vec{v}_n\right)$ \important{spans} $V$.\\
A vector space is called \important{finite-dimensional} if some liste of vectors in it spans the whole space. (A list has finite length by definition).
    \end{definition}
\end{parag}

\begin{parag}{vector}
    \begin{theoreme}
    A list $\left(\vec{v}_1, \ldots, \vec{v}_n\right)$ of vector in $V$ is a basis of $V$ iff every $\vec{v} \in V$ can be written \important{uniquely} in the form
    \begin{align*} 
        \vec{v} = \sum_{i = 1}^{n} \lambda_i \vec{v}_i
    \end{align*}
    \end{theoreme}
    \begin{subparag}{Proof $\implies$}
        The think we say here is, that if there is a basis \important{then} there is a unique representation. The proof is by contradiction:\\
        Suppose that there is two distinct representation:
        \begin{align*} 
            \left(\vec{v}\right) = \sum_{i = 1}^{n} \lambda_i \vec{v}_i - \sum_{i =  1}^{n} \beta_i \vec{v}_i = \vec{0}\\
            = \sum_{i =  1}^{n} \left(\lambda_i - \beta_i\right) \vec{v}_i =  \vec{0}
        \end{align*}
        But all the $\vec{v}_i$ are ``indepedant''therefore all the $\lambda_i - \beta_i$ has to be equal to $0$ therefore, there are equals.
    \end{subparag}
    \begin{subparag}{proof $\impliedby$}
        hm
    \end{subparag}
\end{parag}

\begin{parag}{Span}
   \begin{theoreme}
    Every spanning list in a vector space can be reduced to a basis of the vector space
   \end{theoreme} 
   \begin{subparag}{Proof}
       Remove all the zero-element of the list\\
       Of the new list, remove the second element if it is in the linear span of the first. Repeat the same until we have a list in which the second element is not in the linear span of the first.\\
       Of the new list, remove the third element if it is in the linear span of the first two,\\
       We just continue\\
       At the end,  the result is a list of vector that span the vector space and are lineartly independent (or else one vector can be written as a combination of other vector)\\
       Here we could have use Gram-Schmidt
   \end{subparag}
   \begin{theoreme}
   Any two bases of a finite dimensional vector space have the same length.
   \end{theoreme}
   The \important{dimension} of a finite dimensional vector space, $V$ denoted by dim$\left(v\right)$, is defined to be the length of any basis of $V$.
\end{parag}
\begin{parag}{Few properties of the dimension of a vector space}
    let $V$ be a vector space and suppose that dim$\left(V\right) =  n$:
    \begin{itemize}
        \item if $\left(\vec{v}_1, \ldots, \vec{v}_n\right)$ is a list of linearly independent vectors in $V$ then, it is a basis of $V$.
    \end{itemize}
    
\end{parag}
\begin{parag}{Example}
    We works in $\mathbb{F}_5^3$ and let $S = \{ \vec{v}: \vec{v} = \alpha\left(1, 2, 3\right)\}$\\
    So we can write the vector as:
    \begin{align*} \begin{pmatrix} \alpha \\ 2\alpha \\ 3\alpha \end{pmatrix} = \begin{pmatrix} v_1 \\v_2  \\ v_3 \end{pmatrix}  \end{align*}
    Which implies that:
    \begin{align*}
        \begin{cases}
                2v_1 =  v_2\\
                3v_1 =  v_3
                \end{cases} \iff \begin{cases} 2v_1 + 4v_2 = 0\\ 3v_1 + 4v_3 = 0 \end{cases}
    \end{align*}
    Therefore:
    \begin{align*} 
        S' =  \left\{\vec{v}: 2v_1 + 4v_2 = 0 \text{ and } 3v_1 + 4v_3 = 0\right\}
    \end{align*}
\end{parag}

\begin{parag}{Second key example}
    $v = \mathbb{F}_5^3$ where $\left(v_1, v_2, v_3\right)\left(2\; 3\; 1\right)^T = 0$ which the subspace:
    \begin{align*}
        S =  \left\{\vec{v}: 2v_1 + 3v_2 + v_3 = 0\right\}
    \end{align*}
    The goal is the describe $S$ by a \important{basis}.\\
    Let $v_1 = \alpha$ and $v_2 =  \beta$. where $\alpha, \beta \in \mathbb{F}_5$. Then we know that:
    \begin{align*} v_3 =  -2\alpha - 3\beta =  3\alpha + 2\beta \end{align*}
    We have there two free variables which leads to:
    \begin{align*} 
        \vec{v} &= \left(\alpha, \beta, 3\alpha + 2\beta\right)\\
                &= \alpha g \left(1, 0, 3\right) + \beta\left(0, 1, 2\right)
    \end{align*}
\end{parag}
\begin{parag}{Exercise}
    Let $S$ be the subspace of $\mathbb{F}_7^3$ spanned by $\vec{v} =  \left(4, 3, 1\right)$. Define $S$ by means of equations.
\end{parag}
\begin{parag}{Solution}
    We first see the it is a one dimensional subspace of $\mathbb{F}_7^3$, Therefore, we only need $2$ equations to describe it.\\

\end{parag}
\begin{parag}{Theorem}
    \begin{theoreme}
    The set of solutions in $V = \mathbb{F}^n$ of $m$ linear homogeneous equations in $n$ variables is a subspaces $S$ of $V$.\\
    Let $r$ be the dimensionality of the vector space spanned by the coefficient vectors. Then dim $\left(S\right) =  n-r$.\\
    In particular, if the $m$ vectors of coefficients are linearly independent, then, dim $\left(S\right) ah n-m$.\\
    Conversely, if $S$ is a subspace  of $V =  \mathbb{F}^n$ with dim $\left(S\right) = k$, there exists a set of $n-k$ linear equations with coefficients that form linearly independent vectors in $V$, the solution of which are the vectors in $S$
    \end{theoreme}
\end{parag}


\subsection{Rank of matrix}
\begin{parag}{Definition}
   \begin{definition}
    For any matrix with entries in a field $\mathbb{F}$:
    \begin{itemize}
    \item The dimension of the vector space spanned by its rows
    \item  Equals the dimension of the vector spanned by its columns.
    \end{itemize}
    It is called the \important{rank of the matrix}
   \end{definition} 
\end{parag}

\begin{parag}{theorem 12.2 textbook}
    \begin{theoreme}
    an n-dimensional vector space $V$ over a finite field $\mathbb{F}$:
    \begin{itemize}
        \item is finite,
        \item has cardinality
    \end{itemize}

    
    \begin{align*} \text{card}\left(V\right) =  \left[\text{card}\left(\mathbb{F}\right)\right]^n \end{align*}
    \end{theoreme}
\end{parag}








