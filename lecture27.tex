\subsection{Reed Solomon}
\lecture{27}{2025-05-27}{Reed-Solomon}{}





\begin{parag}{Polynomial}
    
    \begin{definition}
        Let $\mathcal{K}$ be a finite field. A polynomial $P$ with coefficient in $\mathcal{K}$ is a mapping $\mathcal{K} \to \mathcal{K}$ of the form:
        \begin{align*} 
            X \to P\left(X\right) =  a_1 + a_2X + \ldots + a_{m+1} X^m
        \end{align*}
        Where $a_1, \ldots, a_{m+1}$ is a sequence of elements in $\mathcal{K}$.\\
        The power of the polynom is the greatest power affected by a non-zero elements.
    \end{definition}
    For all sequence $\vec{u} =  \left(u_1, \ldots, u_k\right) \in \mathcal{K}^k$ of $k$ elements of $\mathcal{K}$, we called $P_{\vec{u}}$ the polynom where the coefficient are $u_1, \ldots, u_k$ by increasing power. In other terms:
    \begin{definition}
    \begin{align*} 
        P_{\vec{u}}\left(X\right) \underbrace{= }_{def} u_1 + u_2X + u_3X^2 + \cdots + u_kX^{k-1}
    \end{align*}
    Therefore, $P_{\vec{u}}$ is of degree $k-1$. 
    \end{definition}
    We can now define the Reed-Solomon Code:
\end{parag}



\begin{parag}{Reed-Solomon}
   \begin{definition}
   Let $n$ and $k$ be integers with $1 \leq k \leq n$. A Reed-Solomon code with the parameters $\left(n, k\right)$ is defined as:
   \begin{itemize}
       \item The alphabet is a finite field $\mathcal{K}$ of cardinality $\geq n$
       \item Choosen $n$ disctinct elements of $\mathcal{K}$, $a_1, a_2, \ldots, a_n$. A sequence of symbols $\vec{u}=  \left(u_1, \ldots, u_k\right) \in \mathcal{k}^k$ is encoded in the $n$ sequence symbols $\vec{x} = \left(x_1, \ldots, x_n\right) \in \mathcal{K}^n$ defined by:
           \begin{align*} 
               x_i =  P_{\vec{u}}\left(a_i\right) \mathspace \forall i, 1\ldots n
           \end{align*}
   \end{itemize}
   The Reed Solomon $\mathcal{C}$ is the subset of all the encoding $\vec{x}$ possible, for all $\vec{u} \in \mathcal{K}^k$. This is therefore, a codeword of length $n$.
   \end{definition} 
\end{parag}
\begin{parag}{Example}
    Let us build a Reed Solomon code on $\mathbb{F}_5$. Let us choose the maximum possible $n= 5$. (all the elements of $\mathbb{F}_5$).\\
    We also have to choose $5$ elements of $\mathbb{F}_5$, which is all the elements of $\mathbb{F}_5$ ($a_1 = 0, a_2 = 1, \ldots, a_5 = 4$)\\
    Let us compute the encoding. For instance if the message is $\vec{u} = \left(0, 0\right)$ the polynomial to evaluate is $P_{\vec{0}}= 0$, therefore $P_{\vec{0}}\left(a_1\right) =  \cdots = P_{\vec{0}}\left(a_5\right) = 0$, and the codeword corresponding to it is $\vec{x} = \left(0, 0, 0, 0, 0\right)$.\\
    If $\vec{u} =  \left(3, 2\right)$, then $P_{\vec{u}}\left(X\right) = P_{32}\left(X\right) = 3 + 2X$ then:
    \begin{align*} 
        P_{\vec{u}}\left(a_1\right) =  P_{\vec{u}}\left(0\right) = 3\\
        P_{\vec{u}}\left(a_2\right) =  P_{\vec{u}}\left(1\right) = 0\\
        P_{\vec{u}}\left(a_3\right) =  P_{\vec{u}}\left(2\right) = 2\\
        P_{\vec{u}}\left(a_4\right) =  P_{\vec{u}}\left(3\right) = 4\\
        P_{\vec{u}}\left(a_5\right) =  P_{\vec{u}}\left(4\right) = 1
    \end{align*}
    Therefore having $\vec{u} = \left(3, 2\right)$, we get the encoding:
    \begin{align*} 
        \vec{x} =  \left(3, 0, 2, 4, 1\right)
    \end{align*}
    By doing this for the $25$ sequence possible of $k = 2$ elements of $\mathbb{F}_5$ we obtain the table of encoding:
    \begin{center} \begin{tabular}{ccc}$\vec{u}$ & $P_{\vec{u}}\left(X\right)$ & $\vec{x}$ \\ 00 & 0 & 00000 \\ 01 & $X$ & 01234 \\ 02 & $2X$ & 02413 \\ 03 & $3X$ & 03142 \\ 04 & 4 $X$ & 04321 \\ 10 & 1 & 11111 \\ 11 & 1 + X & 12340 \\ 12 & 1 + 2X & 12024 \\ 13 & 1 + 3X & 14203 \\ 14 & 1 + 4X & 10432 \\ 20 & 2 & 22222 \\ 21 & 2 + X & 23401 \\ 22 & 2 + 2X & 24130 \\ 23 & 2 + 3X & 20314 \\ 24 & 2 + 4X & 21043 \\ 30 & 3 & 33333 \\ 31 & 3 + X & 34012 \\ 23 & 3 + 2X & 30241 \\ 34 & 3 + 4X & 32104 \\ 40 & 3 & 44444 \\ 41 & 4 + X & 40123 \\ 42 &4 + 2X  &41302  \\43  &4+3X  & 42031 \\44  & 4 + 4X &43210  \end{tabular} \end{center} 
\end{parag}
\begin{parag}{Property}
    We are going to see now that Reed-Solomon Code have \important{great} property. To be able to remark this, we need a fondamental result of polynomial, which say that a polynomial has a number of root smaller or equal at his degree.
    \begin{theoreme}
        Let $\vec{u} \in \mathcal{K}^k$ where $\mathcal{K}$ is an abelian group. The polynomial $P_{\vec{u}}$ is of degree $k-1$. If there exists $k$ elements all distinct $a_1, \ldots, a_k$ of $\mathcal{K}$ such that $P_{\vec{u}}\left(a_i\right) = 0$, then $\vec{u}= \vec{0}$
    \end{theoreme}
    This is saying that you cannot write your polynomial with more this $\left(x-a\right)$ than k times.
    \begin{subparag}{Proof}
        Here we are doing the proof on finite field which is enough for us (but the real theorem is true even on abelian group $\mathcal{K}$ which is infinite). \\
        Let $\phi$ the mapping $\mathcal{K}^k \to \mathcal{K}^k$ which has $\vec{v} \in \mathcal{K}^k$ abritrary being associated with $\left(P_{\vec{v}}\left(a_1\right), \ldots, P_{\vec{v}}\left(a_k\right)\right)$. The hypothesis is that if $\phi \left(\vec{u}\right) =  \vec{0}$ we want to show that $\vec{u}= \vec{0}$.\\
        To do so, we will show that $\phi$ is injective. By the pigeon hole principle, because the set of départ and the set d'arrivée, are finite and of same cardinality, we juste have to prove that it is surjective, which is what we will do.\\
        Let $\left(x_1, \ldots, x_k\right) \in \mathcal{K}^k$  be an arbitrary vector. Let us look for a polynomial $P$ of degree $\leq k-1$ such that $\left(P\left(a_1\right) = x_1, \ldots, P\left(a_k\right) =  x_k\right)$. Such a problem is called an interpolation: the value of $x_i$ and the point of evaluation $a_i$ are known and we need to find the polynomial d'interpolation of \important{Lagrange}. Let $Q_i, i = 1, \ldots$ the polynomial defined by:
        \begin{align*} 
            Q_1\left(X\right) =  \frac{\left(X-a_1\right)\cdots\left(X-a_{i-1}\right)\left(X-a_{i+1}\right)\cdots\left(X-a_k\right)}{\left(a_i - a_1\right)\cdots\left(a_i - a_{i-1}\right)\left(a_i-a_{i+1}\right)\cdots\left(a_i-a_k\right)}
        \end{align*}
        \begin{framedremark}
        Here what we are doing is skipping the $a_i$ coefficient and putting all the zero possible for all other $a_n$. This is the reason why for every element except $a_i$ $Q_i\left(X\right) = 0$.\\
        The reason why it is $1$ when $X = a_i$ is:\\
        If you take the polynomial at $a_i$, you get:
        \begin{align*} 
            Q_1\left(X\right) =  \frac{\left(a_i-a_1\right)\cdots\left(a_i-a_{i-1}\right)\left(a_i -a_{i+1}\right)\cdots\left(a_i-a_k\right)}{\left(a_i - a_1\right)\cdots\left(a_i - a_{i-1}\right)\left(a_i-a_{i+1}\right)\cdots\left(a_i-a_k\right)} = 1
        \end{align*}
        \end{framedremark}
        This is a polynomial of degree $\leq k-1$ (the product of $k-1$ termes of degree $1$) and it has the property that $Q_i\left(a_i\right)= 1$ and $Q_i\left(a_j\right) = 0$ for $i \neq j$.\\
        Soit maintenant $P$ be the polynomial defined by:
        \begin{align*} 
            P\left(X\right) = x_1Q_1\left(x\right) + \cdots + x_kQ_k\left(X\right)
        \end{align*}
        Such that $P$ is a polynomial of degree $\leq k-1$ (because we are doing the sum of all the polynomial of degree $k-1$).\\
        Let $v_1 = $ the coefficient of degree $0$ of $P$, etc $\ldots$ $v_k = $the coefficient of de degree $k-1$ of $P$, such that $P = P_{\vec{v}}$ and then $\psi\left(\vec{v}\right) =  \vec{x}$.\\
        We have then shown that $\psi$ is surjective.\\
        Therefore, $\phi$ is injective and $\phi\left(\vec{0}\right) = \vec{0}$. Finally, $\vec{u}$ and $\vec{0}$ has the same image by $\phi$ which leads that $\vec{u} =  \vec{0}$.
    \end{subparag}
\end{parag}



\begin{parag}{Linearity of Reed-Solomon codes:}
    In fact, the reed Solomon code are linear codes:
    \begin{theoreme}
        A Reed-Solomon codes of parameters $\left(n, k\right)$ is a linear blockcode of length $n$ and dimension $k$.
    \end{theoreme}
    
    \begin{framedremark}
    In order to prove this, we have to show that the subspace $\mathcal{C}$ is a vector space of $\mathcal{C}^n$
    \end{framedremark}
    \begin{subparag}{Proof}
        To prove this we will do as we did in linear algebra:\\
            Let $\vec{u}, \vec{v} \in \mathbb{F}^k$ and $\alpha \in \mathbb{F}$.Notice that:
            \begin{align*} 
                P_{\alpha \vec{u}}\left(x\right) =  \alpha u_1 + \alpha u_2x + \cdots + \alpha u_kx^{k-1}= \alpha P_{\vec{u}}\left(x\right)
            \end{align*}
            For the second clause:
            \begin{align*} 
                P_{\vec{u} + \vec{v}}\left(x\right) = \left(u_1 + v_1\right) + \left(u_2 + v_2\right)x + \cdots + \left(u_k + v_k\right)x^{k-1} = P_{\vec{u}}\left(x\right) + P_{\vec{v}}\left(x\right)
            \end{align*}
            Hence:
            \begin{align*} 
                P_{\alpha \vec{u} + \vec{v}}\left(x\right) =  \alpha P_{\vec{u}}\left(x\right) + P_{\vec{v}}\left(x\right)
            \end{align*}
            Then, is we have the following mapping
            \begin{align*} 
                \vec{u} \to \vec{x}  \in \mathcal{C}\\
                \vec{v} \to \vec{y} \in \mathcal{C}
            \end{align*}
            Then we can say that 
            \begin{align*} \alpha \vec{u} + \vec{v} \to \alpha \vec{x} + \vec{y} \in \mathcal{C} \end{align*}
    \end{subparag}
\end{parag}
\begin{parag}{Design-parameters $k$ }
    \begin{theoreme}
        The design parameter $k$ is indeed the dimension of the RS-code.
    \end{theoreme}
    \begin{subparag}{Proof}
        It suffices to prove that the map is injective (one to one) implying that there are $\left[\text{card}\left(\mathbb{F}\right)\right]^k$ disctinct codewords, hence that $k$ is the dimension of the code.\\
        When we proved the fundamental theorem of algebra, we showed that the map:
        \begin{align*} 
            \psi: \mathbb{F}^k \to \mathbb{F}^k\\
            \vec{u} \to \left(P_{\vec{u}}\left(a_{1}\right), \ldots, P_{\vec{u}}\left(a_k\right)\right)
        \end{align*}
        is one to one.\\
        This garantese the encoding map:
        \begin{align*}  
            \mathbb{F}^k \to \mathcal{C} \\
            \vec{u} \to \left(P_{\vec{u}}\left(a_i\right), \ldots, P_{\vec{u}}\left(a_k\right), P_{\vec{u}}\left(a_{k+1}\right), \ldots, P_{\vec{u}}\left(a_n\right)\right)
        \end{align*}
    is one to one
    \end{subparag}
    
\end{parag}
\begin{parag}{Reed solomon Code are MDS}
    \begin{theoreme}
        Reed Solomon code are MDS
    \end{theoreme}
    \begin{subparag}{proof}
        We want to prove that $d_{min} =  n- k + 1$:
        \begin{itemize}
            \item Let $\vec{u} \in \mathbb{F}^k$ be a non zero information vector. $P_{\vec{u}}\left(x\right)$ is a non zero polynomial of degree at most $k-1$. Hence it has at most distinct $k-1$ roots.
            \item The corresponding codeword $\vec{c}$ (obtained by evaluating $P_{\vec{u}}\left(x\right)$ at $n$ distince values) has at most $k-1$ zeros
            \item Hence $w\left(\vec{c}\right) \geq n \left(k-1\right) = n-k + 1 \mathspace \forall \vec{c}$ 
            \item Since $\vec{c}$ is an arbitrary non zero codeword and RS code are linear code
                \begin{align*} d_{min} \leq n - k  + 1 \end{align*}
            \item By the Singleton's bound, $d_{min} \leq n - k + 1$
        \end{itemize}
        Hence $d_{min} =  n-k + 1$
    \end{subparag}
\end{parag}
\subsubsection{Reed Solomon is goated}
\begin{parag}{Theorem Textbook 14.3}
    \begin{theoreme}
    A Reed Solomon code with design parameters $k$ and $n$ is linear $\left(n, k\right)$ code of minimum distance $d_{min} = n-k+1$, i.e. it attains singleton's bound with equality
    \end{theoreme}
    \begin{framedremark}
    Note that the condition:
    \begin{align*} \left|\mathbb{F}\right| \geq n \end{align*}
    Is necessary or else we can't find $n$ discting field elements $a_1, \ldots, a_n$.
    \end{framedremark}
\end{parag}



\begin{parag}{Exercise}
    Find a generator matrix $G$ for this code over $\mathbb{F}_3$.\\
    Here because we are in $F_{3}$ we know that $n = 3$ to be able to have $a_1, a_2, a_3$ distinct elements ($\left|\mathbb{F}_3\right| = 3$), we then take $k = 2$. We have here $9$ distincts vectors and each has a mapping like this:

    \begin{center} \begin{tabular}{c|c|c}$\vec{u}$ & $P_{\vec{u}}\left(x\right)$ & $\vec{c}$ \\ \hline \hline 00 & 0 & 000 \\ \hline 01 & x & 012  \\ \hline
     02 & 2x & 021 \\ 1\hline
0 & 1 & 111 \\ \hline
11 & 1 + x & 120 \\ \hline
12  & 1 + 2x & 102 \\ \hline
20  & 2 & 222 \\ \hline
21 & 2 + x & 210  \\\hline
     22 & 2 + 2x & 210 \end{tabular} \end{center} 
\end{parag}

To make the matrix $G$ we need to find two codewords that are linearly independent\\
Here are a few choices:
\begin{align*} 
    G = \begin{pmatrix} 0 & 1 & 1 \\ 1 & 1 & 1 \end{pmatrix} \mathspace G = \begin{pmatrix} 1 & 2 & 0 \\ 2 & 0 & 1 \end{pmatrix} \mathspace G = \begin{pmatrix} 2 & 1 & 0 \\ 1 & 0 & 2 \end{pmatrix} 
\end{align*}
We just recall that each generator matrix defined an input output map.\\
The map that we originally used to define $RS$ coes:
\begin{align*} \vec{u} \to P_{\vec{u}}\left(x\right) \to \vec{c} =  P_{\vec{u}}\left(a_1\right), \ldots, P_{\vec{u}}\left(a_n\right) \end{align*}
is just one of many possible maps. (each $G$ gives a different mapping (like $\vec{u}G_1 \neq \vec{u}G_2$ but still in the same $\mathcal{C}$).\\
In fact there are $\left(q^k-1\right)\left(q^k-q\right)\cdots\left(q^k - q^{k-1}\right)$ maps for a linear code of dimension $k$ over $\mathbb{F}_q$.
\end{parag}
\begin{parag}{Exercise}
    Which of the following is a valid parity check matrix? (for the same code)
    \begin{itemize}
        \item $A =  \begin{pmatrix} 1 & 1 & 1 \\ 0 & 1 & 2 \end{pmatrix} $
        \item $B =  \begin{pmatrix} 0 & 1 & 2 \\ 1 & 0 & 2 \end{pmatrix} $
        \item $C =  \begin{pmatrix} 1 & 1 & 1 \end{pmatrix} $
    \end{itemize}
\begin{subparag}{Solution}
    First we know that the parity check matrix is of dimension $n-k \times n$, which here is $3-2 \times 3$ which can only be the second.\\
    Moreover, we can check also with the definition of the codeword like this:
    \begin{align*} \vec{y} \in \mathcal{C} \iff \vec{y}H^T =  \vec{0} \end{align*}
\end{subparag}
Now if we wanted to have the generator matrix that give s the same result as the encoding map of the table how can we do it?\\
The easiest way is to check for two vectors $\left(1, 0\right)$ and $\left(0, 1\right)$ and put their encoding in the rows of  $G$:
\begin{align*} 
    \left(1, 0\right)G =  \left(1, 1, 1\right)\\
    \left(0, 1\right)G = \left(0, 1, 2\right)
\end{align*}
Which leads to:
\begin{align*} 
    G =  \begin{pmatrix} 1 & 1 & 1 \\ 0 & 1 & 2 \end{pmatrix} 
\end{align*}
\end{parag}
\begin{parag}{More generally}
    If we wanted the generator matrix for a given encoding map (like we just did), it suffices to find the codeword that correspond to the following length $k$ information words:
    \begin{align*} 
        \left(1, 0, 0, &\ldots, 0, 0\right)\\
        \left(0, 1 0, &\ldots, 0, 0\right)\\
        \left(0, 0, 1, &\ldots, 0, 0\right)\\
        &\vdots \\
        \left(0, 0, 0, &\ldots, 1, 0\right)\\
        \left(0, 0, 0, &\ldots, 0, 1\right)\\
    \end{align*}
    The corresponding codewords are linearly independent and are $k$ in number. Hence they form a basis.
    \begin{subparag}{Proof of linearly independent}
        We want to prove that a linear encoding map sends linearly independent information vectors to linearly independent codewords (linearly independent words $\to$ linearly independent codewords)\\
        Let $\vec{u}_i \in \mathbb{F}^k, i = 1, \ldots, k$ be a collection of linearly independent information vectors and let $\vec{c}_i\in \mathbb{F}^n$ be the corresponding codewords.\\
        We prove that the codewords are linearly independent.\\
        Suppose , to the contrary that the codewords are linearly dependent, i.e.,
        \begin{align*} \sum_{i =  1}^{k} \lambda_i\vec{u}_i = 0 \end{align*}
        Where there are at least a $\lambda_i \neq 0$, Then:
        \begin{align*} 
            \sum_{i = 1}^{k} \lambda_i\vec{u}_i = 0
        \end{align*}
        This is true because we know that the encoding map is \important{injective} and injective implies that there are only one element that goes to $0$ and it is $0$\\
        Which contradicts the assumption that the information vectors are linearly independent. In particular, the map that we used to define the code sends $\vec{u} \in \mathbb{F}^k$ to:
        \begin{align*} 
            \vec{c} =  \left(P_{\vec{u}}\left(a_1\right), \ldots, P_{\vec{u}}\left(a_n\right)\right)
        \end{align*}
        if $\vec{u}$ has $1$ at position $i$ and $0$ elsewhere, then $P_{\vec{u}}\left(x\right) =  x^{i-1}$\\
        Hence, the map the we have used to defined RS codes has the generator matrix:
        \begin{align*} 
            G =  \begin{pmatrix} 1 & 1 & 1 & \ldots & 1 \\ a_1 & a_2 & a_3 & \ldots & a_n \\ \left(a_1\right)^2 & \left(a_2\right)^2 & \left(a_3\right)^2 & \ldots & \left(a_n\right)^2 \\ \vdots & \vdots & \vdots & \ddots & \vdots\\ \left(a_1\right)^{k-1} & \left(a_2\right)^{k-1} & \left(a_3\right)^{k-1} & \cdots & \left(a_n\right)^{k-1}
            \end{pmatrix} 
        \end{align*}
    \end{subparag}
\end{parag}
\begin{parag}{Example}
    CDs use cross Interleaved Reed Solomon Codes (CIRC) as follows:
    \begin{itemize}
        \item Each byte of the source is seen as an element of $\mathbb{F} =  \mathbb{F}_{2^8}$
        \item Source symbols, are encoded using a $\left(28, 24\right)$ RS code over $\mathbb{F}$
        \item The elements of a sequence of many codewords are interleaved
        \item The result is encoded using a $\left(32, 28\right)$ RS code over $\mathbb{F}$
    \end{itemize}
    
    Notice that both codes have $d_{min} = 5$, but this small distance is compensated by the interleaver which distribute strings of error among many codewords.\\
    The end result is that error bursts up to 4000 bits can be corrected. We have roughly that many bits in a track segment of length 2.5mm
\end{parag}

\subsubsection{Summary of chapter 3}
\begin{parag}{Linear code: basis properties, Design, Decoding}
    \begin{itemize}
        \item Finite field: A finite set with two operations (sum and product) satisfying the natural properties (as you know them over the reals)
            \begin{itemize}
                \item Only exists if the cardinality of the finite set is a prime power, card$\left(\mathbb{F}\right) =  p^m$
            \end{itemize}
        \item Vectors spaces over finite field. Subspaces, Basis
        \item Linear code = Subspace of a vector space over a finite field
        \item Generator matrix (= collection of vectors spanning the subspace)
            \begin{itemize}
                \item Particularly desirable form: Systematic generator matrix
            \end{itemize}
        \item Parity check matrix
        \item Key example: Hamming codes
    \end{itemize}
\end{parag}
\begin{parag}{Reed-Solomon Codes}
    \begin{itemize}
        \item A very popular class of linear codes over a finite field $\mathbb{F}$, They are MDS
        \item Block length has to be $n \leq \left|\mathbb{F}\right|$ (so we need large finite fields)
        \item They can be neatly described via polynomials
    \end{itemize}
\end{parag}















