\lecture{28}{2025-05-28}{Last lecture}{}
   
   
 I juste wanted to make it juste clearer what is lagrange's interpolation polynomials (because in the book it isn't really mentionned):
 \subsubsection{Lagrange's interpolation polynomials}
 \begin{parag}{Interpolation via polynomials}
     \begin{subparag}{Problem}
         Given a field $\mathbb{F}$ and $k$ pairs $\left(a_i, y_i\right)\in \mathbb{F}^2$, where the $a_i$ are all distinct, is there a polynomial $P\left(X\right)$ over $\mathbb{F}$ of degree at most $k-1$ (hence described by at most $k$ coefficients) such that:
         \begin{align*} 
             P\left(a_i\right) =  y_i, \mathspace i = 1, \ldots k
         \end{align*}
     \end{subparag}
     The answer is yes, and we obtain those via Lagrange's interpolation polynomials.\\
     This is the same process that we explained in the proof of yeasterday (in the remark). For instance let us take some examples:
     \begin{subparag}{Example}
         We first fix a field $\mathbb{F}$ and disctinct field elements $a_1, a_2, a_3$ as well as $y_1, y_2, y_3  $ (not necessarily distinct).\\
         We are looking for a polynomial $P\left(X\right)$ of degree at most $2$ ($k-1 $) and coefficients in $\mathbb{F}$ such that $P\left(a_i\right) =  y_i$.\\
         Suppose we can find a polynomial $Q_1\left(x\right)$ of degree at most $2$, such that:
         \begin{align*} 
             Q_1\left(x\right) =  
             \begin{cases}
                 1, \mathspace x = a_1\\
                 0, \mathspace x =  a_i \neq a_1
             \end{cases}
         \end{align*}
         (which is the same thing for our $Q_i\left(X\right)$ before)\\
         Now we suppose that $Q_2\left(x\right)$ behaves the same as for $a_2$ and the same for $Q_3\left(x\right)$ and $a_3$.\\
         Then the desired polynomial is
         \begin{align*} P\left(X\right) = y_1Q_1\left(x\right) + y_2Q_2\left(x\right) + y_3Q_3\left(x\right) \end{align*}
         Now if we want to construct the first $Q\left(X\right)$ this is then:
         \begin{align*} 
             Q_1\left(X\right) = \frac{\left(x - a_2\right)\left(x-a_3\right)}{\left(a_1 - a_2\right)\left(a_1 - a_3\right)}
         \end{align*}
         We do the same for the two other $Q_i$.
     \end{subparag}
     \begin{subparag}{Concrete example}
         Over $\mathbb{F}_5$ find the polynomial $P\left(X\right)$ of degree not xceeding $2$ for which $P\left(a_i\right) =  y_i$ and this for:
         \begin{center} \begin{tabular}{cc}$i$ & $\left(a_i, y_i\right)$ \\ 1 & $\left(2, 3\right)$ \\ 2 & $\left(1, 0\right)$ \\ 3 & $\left(0, 4\right)$ \end{tabular} \end{center} 
         To do so, we then will use the same process as before:
         \begin{align*} 
             P\left(x\right) = y_1Q_1\left(x\right) + y_2Q_2\left(x\right) + y_3Q_3\left(x\right)
         \end{align*}
         We then construct the $Q_i$:
         \begin{align*} 
             P\left(x\right) &= 3\left(\frac{\left(x-1\right)x}{\left(2-1\right)2}\right) + 0 \cdot  \cdots + 4\frac{\left(x-2\right)\left(x-1\right)}{\left(4-2\right)\left(4-1\right)}\\
             &= 3\left(x-1\right)x + 3\left(x-2\right)\left(x_1\right)\\
             &= 4x^2 - 4x + 2x^2 -6x + 4\\
             &= x^2 + 4
         \end{align*}
     \end{subparag}
 \end{parag}
 \begin{parag}{Conclusion}
     It should be obvious from the above example that we can proceed similarly for any field $\mathbb{F}$ for any positive integer $k$, and for any given set of $k$ points with components in $\mathbb{F}$.
 \end{parag}
 \begin{parag}{Fundamental theorem of Algebra}
     I already wrote this before but too much is better than not enough:
     \begin{theoreme}
     Let $P\left(x\right)$ be a polynomial of degree at most $k-1$ over a field. if $P\left(x\right) \neq 0$ then the number of its distinct roots is at most $k-1$.
     \end{theoreme}
     \begin{subparag}{Example}
         You can see this it for polynomial of degree $1$: $ax + b$ this is already a ``root'' so the number is one.\\
         For a polynomial of degree $2$ we can go up to two roots, three for cubic polynomial...
     \end{subparag}
     \begin{framedremark}
     I already done the proof above in lecture 26.
     \end{framedremark}
 \end{parag}
 
 
 
 
   
