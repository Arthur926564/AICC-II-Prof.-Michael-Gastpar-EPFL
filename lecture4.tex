\lecture{4}{2025-02-26}{Continue}{}


\begin{parag}{Key observation}
    The right hand side of:
    \begin{align*}
        L(S, \Gamma) = \sum_{s \in \mathcal{A}} p(s) l ( \Gamma (s))
    \end{align*}
    \begin{align*}
        H_D(S) = \sum_{s \in \mathcal{A}} p(s) \log_D \frac{1}{p_S(s)}
    \end{align*}
    are identical if $l( \Gamma (s))$
    \begin{itemize}
        \item Unfortunately $l( \Gamma (s)) = \log_D \frac{1}{p_S(s)}$ is often not possible (not an integer)
        \item How about choosing
    \end{itemize}
    
    \begin{theoreme}
        \begin{itemize}
            \item For every random variable $S \in \mathcal{A}$
        \end{itemize}
    \end{theoreme}
\end{parag}

\begin{parag}{Theorem}
    
    \begin{theoreme}
        The average codeword length of a D-ary Shannon-Fano code for the random variable $S$ fulfils: 
        \begin{align*}
            H_D(S) \leq L(S, \Gamma_{SF}) < H_D(S) + 1
        \end{align*}
        
    \end{theoreme}
   \begin{subparag}{Proof}
       it suffices to prove the upper bound (we have already proved the lower bound) 
       \\
       First suppose that we could use $l_i = -\log p_i$. The average length would be:
       \begin{align*}
           L(S, \Gamma) = \sum_i p_i l_i = \sum_i p_i (-\log_D p_i) = H_D(S)
       \end{align*}
       Instead we use $l_i = \left\lceil -\log p_i \right \rceil < - \log p_1 + 1 $
       
   \end{subparag} 

\end{parag}

